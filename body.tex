\section{Introduction}

Justification for having Scheduler on LSST, reference to requirements and main telescope and site paper. 

Introduction to the concept of a component in the LSST Observatory Control System. 

Description of the Scheduler component and its role during operations/observations. 

\section{Telemetry}

Describes the telemetry available to the scheduler. How does the scheduler gets telemetry data? 
Where each telemetry is coming from?

\subsection{Time}

Time is one of the most basic and crucial information the Scheduler needs to be able to reliably work. As all 
high-level software components, the Scheduler runs with time synchronization with the observatory clock. 

\subsection{Seeing}

The observatory is equipped with a DIMM that publishes measurements to the observatory middleware and is stored in 
the EFD. Data from more than one equipment can be available at any single time to the Scheduler. In addition, information
about image quality will also be available to Scheduler as data is processed. 

Usually we will be interested in more than just the last measurement taken from the DIMM instruments, as time trend analysis 
is important for the Scheduler to perform predicted scheduling. 

After querying the data from the EFD, the Scheduler sorts the data according to a priority rule and passes it to the 

\subsection{Cloud cover}

\subsection{Downtime}

\subsection{Observatory State}

\subsection{Observing Queue}

\section{Models}

Describes the general concept of models in the Scheduler framework and how they can be expanded/added. Describe 
the currently available models. 

\subsection{Seeing model}

\subsection{Cloud model}

\subsection{Sky model}

\subsection{Downtime model}

\subsection{Observatory model}

\section{Scheduling Algorithm}

Describe the concept of the scheduling algorithm. What are the assumptions made by the scheduler regarding the underlying 
algorithm, description of the interface. 

\section{Output and Operation modes}

Describe the different modes which the Scheduler can be operated and what are the output data that the scheduler 
generates. 

\section{Conclusions}
