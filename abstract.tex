
\begin{abstract}

This paper will give a general overview of the Scheduler component. The Scheduler is designed to provide autonomous 
observing capability to the observatory. For such, it is fed with a myriad of telemetry about the state of the observatory (e.g. 
telescope, dome, instrument state), weather information (e.g. seeing, wind, cloud cover) and so on. The Scheduler also
contains a set of base models that provides additional information based on the input telemetry. The information gathered 
by the Scheduler is then passed over to a scheduling algorithm that, in turn, returns an observing target. The scheduling 
algorithm itself is not part of the Scheduler, which only specify a set of rules that a an algorithm must follow so it can be
used. Here we will describe the telemetry information available to the Scheduler, how that telemetry data is gathered and
how it is made available to the scheduling algorithm. The same will also be done for the currently available models. 
Finally we will define the scheduling algorithm interface and describe how the Scheduler operates in the different modes. 
Details about the adopted scheduling algorithm will be part of a separate document. 

\end{abstract}

